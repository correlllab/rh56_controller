\documentclass[11pt,a4paper]{article}
\usepackage{amsmath,amssymb,bm}
\usepackage{geometry}
\usepackage{hyperref}
\usepackage{booktabs}
\usepackage{xcolor}
\geometry{margin=2.5cm}

\newcommand{\R}{\mathbf{R}}
\newcommand{\p}{\mathbf{p}}
\newcommand{\T}{\mathbf{T}}
\newcommand{\d}{\mathbf{d}}
\newcommand{\e}{\mathbf{e}}
\newcommand{\bx}{\hat{\mathbf{x}}}
\newcommand{\by}{\hat{\mathbf{y}}}
\newcommand{\bz}{\hat{\mathbf{z}}}

\title{Antipodal Grasp Geometry for the Inspire RH56 Dexterous Hand\\
\large Mathematical Reference}
\author{rh56\_controller}
\date{}

\begin{document}
\maketitle
\tableofcontents
\bigskip

% ---------------------------------------------------------------------------
\section{Coordinate Systems}

\subsection{Hand Base Frame $\mathcal{F}_b$}

The origin is the root of the floating hand model.  Axes are defined by the
body quaternions in \texttt{inspire\_right.xml}:
\begin{align*}
  \bx_b &: \text{palm closure direction (fingertips curl toward $+X$ when flexed)}\\
  \by_b &: \text{finger spread (index at $Y\approx+32\,\text{mm}$, pinky at $Y\approx-26\,\text{mm}$)}\\
  \bz_b &: \text{finger extension (wrist toward fingertip when open)}
\end{align*}
At full extension all non-thumb fingertip $Z$ values lie in $[200,\,216]\,\text{mm}$.
At full flexion they lie in $[90,\,100]\,\text{mm}$, with $X$ advancing to $[27,\,34]\,\text{mm}$.

\subsection{World Frame $\mathcal{F}_w$}

For a top-down grasp approach the hand hangs fingers-downward above the target plane
at world height $g_z$.  The combined rotation
\begin{equation}
  \R(\theta) \;=\; R_y(\theta)\,R_x(\pi)
  \label{eq:R}
\end{equation}
maps $\mathcal{F}_b \to \mathcal{F}_w$.  Written explicitly:
\begin{equation}
  \R(\theta) = \begin{pmatrix}
    \cos\theta & 0 & -\sin\theta \\
    0 & -1 & 0 \\
    -\sin\theta & 0 & -\cos\theta
  \end{pmatrix}
  \label{eq:Rmat}
\end{equation}
$R_x(\pi)$ alone maps $\bz_b \to -\bz_w$ (fingers down) and $\bx_b \to \bx_w$
(closure stays horizontal).  The subsequent $R_y(\theta)$ tilts the hand about the
spread axis to level the grasp plane.

% ---------------------------------------------------------------------------
\section{Forward Kinematics Model}

\subsection{Joint Coupling}

Equality constraints in \texttt{inspire\_right.xml} use MuJoCo's \texttt{polycoef}
convention: \texttt{polycoef="$a_0\;a_1$"} enforces $q_1 = a_0 + a_1\,q_2$.

\begin{table}[h]
\centering
\begin{tabular}{lll}
\toprule
Joint pair & Coupling equation & Source \\
\midrule
pinky / ring / middle intermediate & $q_\text{inter} = q_\text{prox}$ & \texttt{polycoef="0 1 0 0 0"} \\
index intermediate & $q_\text{inter} = 0.15 + q_\text{prox}$ & \texttt{polycoef="0.15 1 0 0 0"} \\
thumb intermediate & $q_\text{inter} = 0.15 + 1.25\,q_\text{pitch}$ & \texttt{polycoef="0.15 1.25 0 0"} \\
thumb distal & $q_\text{dist}   = 0.15 + 0.75\,q_\text{pitch}$ & \texttt{polycoef="0.15 0.75 0 0 0"} \\
\bottomrule
\end{tabular}
\caption{Joint coupling equations. The 0.15~rad preload means all coupled joints are
pre-bent $\approx8.6°$ at zero proximal angle, shifting fingertip positions relative
to a na\"ive rigid-link model.}
\end{table}

\subsection{FK Sweep}

Because the 3-D body quaternions in the XML make analytical FK intractable, all
fingertip positions are computed numerically via MuJoCo's \texttt{mj\_kinematics}.

\paragraph{Non-thumb fingers.}
For finger $f$ with maximum control $c_f^{\max}$, sweep 200 uniformly-spaced
control values $c \in [0, c_f^{\max}]$:
\begin{enumerate}
  \item Zero all \texttt{qpos}.
  \item Set $q_\text{prox}(f) = c$ and coupled joint(s) per Table~1.
  \item Call \texttt{mj\_kinematics} (kinematics only; no dynamics, no equality solver).
  \item Record $\p_f(c) = \texttt{site\_xpos[tip\_site]}$ in the base frame.
\end{enumerate}
The result is a table $\{c_i, \p_f(c_i)\}_{i=1}^{200}$ interpolated with
\texttt{scipy.interpolate.interp1d} (linear, extrapolation to boundary).

\paragraph{Thumb.}
The thumb has two independent DOF (pitch $p \in [0,\,0.6]$ and yaw
$y \in [0,\,1.308]$).  A $50 \times 50$ grid sweep produces
$\T(p, y) \in \mathbb{R}^{50\times50\times3}$, interpolated with
\texttt{RegularGridInterpolator}.

\paragraph{Caching.}
Tables are saved to \texttt{.fk\_cache.npz} and reloaded on subsequent runs.
Rebuild is forced with \texttt{InspireHandFK(rebuild=True)}.

% ---------------------------------------------------------------------------
\section{Grasp Tilt}

\subsection{The Coplanarity Constraint}

Let $\d = \T - \mathbf{C}$ be the vector from the reference finger tip (or centroid)
to the thumb tip, all in the base frame.  The grasp plane is horizontal in world frame
when the world-$Z$ components of $\T$ and $\mathbf{C}$ are equal, i.e.\ when
\begin{equation}
  \bigl(\R(\theta)\,\d\bigr)_z = 0.
\end{equation}
Using \eqref{eq:Rmat}:
\begin{equation}
  -\sin\theta\,d_x - \cos\theta\,d_z = 0
  \;\implies\;
  \tan\theta = -\frac{d_z}{d_x}
  \;\implies\;
  \boxed{\theta = \operatorname{atan2}(-d_z,\; d_x).}
  \label{eq:tilt}
\end{equation}

\subsection{Effective Grasp Width Identity}

Substituting \eqref{eq:tilt} into the world-$X$ component:
\begin{equation}
  \bigl(\R(\theta)\,\d\bigr)_x
  = \cos\theta\,d_x - \sin\theta\,d_z
  = \frac{d_x^2}{r} + \frac{d_z^2}{r}
  = r
  \qquad\text{where } r = \sqrt{d_x^2 + d_z^2}.
  \label{eq:width}
\end{equation}
The graspable width equals the \textbf{XZ-plane distance} between thumb tip and finger
centroid in the base frame.  The $Y$ component (anatomical spread, $\approx70$~mm for
thumb vs.\ index) is not closeable by rotation and does not contribute.

\textit{Corollary}: using the full 3-D distance as the width criterion causes $d_z$ to
change sign at the minimum distance, which flips the sign of $\theta$ and creates a
~155° discontinuity.  The XZ distance avoids this entirely.

\subsection{Modal Shift at $d_x = 0$ and the Clip Fix}

As the closure parameter $s$ increases toward maximum, the thumb tip eventually reaches
the same $X$ as the reference finger ($d_x = 0$ at $s = s_{d_0} \approx 0.437$ for
index), after which $d_x < 0$ (thumb behind finger in $X$).  From \eqref{eq:tilt}:
\begin{equation}
  d_x > 0: \quad \theta \in (-\tfrac\pi2, \tfrac\pi2)
  \quad\text{(standard top-down approach)}
\end{equation}
\begin{equation}
  d_x = 0: \quad \theta = \tfrac\pi2 \quad\text{exactly (width } \approx 14.1\,\text{mm)}
\end{equation}
\begin{equation}
  d_x < 0: \quad |\theta| > \tfrac\pi2
  \quad\text{(hand tilted past horizontal, clip activates)}
\end{equation}

The old normalization $\theta \mathrel{-}= \pi$ when $\theta > \pi/2$ creates a
\emph{discontinuous} 155° jump.  The correct fix is to \textbf{clip}:
\begin{equation}
  \theta_\text{applied} = \operatorname{clip}\!\bigl(\theta,\;-\tfrac\pi2,\;\tfrac\pi2\bigr).
  \label{eq:clip}
\end{equation}
This produces a smooth \textbf{modal shift}: for $d_x \leq 0$, the hand holds a fixed
90° side-approach orientation and the residual world-$Z$ spread of the tips is
$|d_x| \lesssim 5$~mm, which is acceptable.  Tilt increases monotonically toward 90°
as width decreases to $\approx14.1$~mm (index), then freezes there for
the remaining $8.7$–$14.1$~mm range.

% ---------------------------------------------------------------------------
\section{Attempted Improvements and Why They Failed}

The narrow-width regime produces a large, fixed tilt ($\theta = 90°$) and causes the
thumb's kinematic links to drop below the grasp plane in the base frame.
Three strategies were explored to improve this.

\subsection{Blend Ramp}

\paragraph{Idea.}
Precompute $s_{d_0}$ (where $d_x = 0$) and $s_{\min}$ (XZ minimum).
For $s \in [s_{d_0}, s_{\min}]$ blend tilt linearly from $\pi/2$ to a precomputed
$\theta_\text{target} \approx -35°$ while keeping ctrl values proportional.

\paragraph{Failure.}
The tilt ramp traverses $90° \to -35°$ in $\approx5$~mm of width change, causing a
visible orientation flip as the slider moves.  The root cause is that the
\emph{exact} coplanarity tilt in the narrow regime lives on the upper branch
$\theta \in (90°, 148°)$, which is physically inaccessible for a top-down grasp.
Any smooth approximation must traverse a large angular distance.

\subsection{2-D Minimum-Tilt Search}

\paragraph{Idea.}
At each target width, sweep \texttt{ctrl\_pitch} over 50 values; for each, use Brent's
method to find the \texttt{ctrl\_ref} achieving the target width; select the pair
minimising $|\theta|$.  Ctrl values are decoupled from the proportional line.

\paragraph{Failure.}
Severe jitter: large discontinuous jumps in \texttt{ctrl\_pitch} and \texttt{ctrl\_ref}
at adjacent widths.  The root cause is that the argmin of the tilt landscape over the
2-D ctrl space is a non-convex, wide-flat function whose global minimiser jumps
discontinuously as the width constraint changes by even 1~mm.
Searching independently at each width destroys the global smoothness that the
proportional parameterisation provides.

\subsection{Precomputed Coplanar Table}

\paragraph{Idea.}
Sweep \texttt{ctrl\_ref} monotonically; for each, find the coplanar \texttt{ctrl\_pitch}
such that $T_z = I_z$ (zero tilt) via \texttt{thumb\_tip\_at\_z}.  Store as
\texttt{interp1d} tables; replace the runtime search with an $O(1)$ lookup.

\paragraph{Failure (anticipated).}
The coplanar trajectory replaces the full proportional trajectory for \emph{all} widths,
fundamentally changing hand pose even at wide widths where the original was already good.
Continuity at the wide-to-narrow transition was not guaranteed.

\subsection{Key Lesson}

The proportional parameterisation
\begin{equation}
  c_\text{pitch}(s) = s\,c_\text{pitch}^{\max},\qquad
  c_\text{ref}(s) = s\,c_\text{ref}^{\max}
\end{equation}
works because it is a \emph{global} 1-D smooth curve through a well-behaved region
of the 2-D ctrl space.  Any strategy that re-optimises the 2-D ctrl pair independently
at each width will encounter the non-convex, discontinuous landscape and reproduce
the jitter.  The only safe improvement strategy is a small, smooth, globally-defined
perturbation of the proportional trajectory that activates only when $d_x < \varepsilon$.

% ---------------------------------------------------------------------------
\section{Closure Solver}

\subsection{Uniform Closure Parameter}

For precision grasps a single scalar $s \in [0,1]$ parameterises both the thumb and
reference-finger motion proportionally to their respective maximum control values:
\begin{equation}
  c_\text{pitch}(s) = s\,c_\text{pitch}^{\max}, \qquad
  c_\text{ref}(s)   = s\,c_\text{ref}^{\max}.
  \label{eq:s}
\end{equation}
The XZ-plane distance at closure $s$ is
\begin{equation}
  D(s) = \bigl\|\bigl(\T(s) - \mathbf{I}(s)\bigr)_{\{x,z\}}\bigr\|_2
       = \sqrt{\bigl(T_x - I_x\bigr)^2 + \bigl(T_z - I_z\bigr)^2}.
\end{equation}
$D(s)$ is monotone-decreasing on $[0, s_{\min}]$ where $s_{\min} = \arg\min_s D(s)$.
The target width $W$ is achieved by solving
\begin{equation}
  D(s) - W = 0
  \label{eq:brentq}
\end{equation}
on $[0, s_{\min}]$ using Brent's method (\texttt{scipy.optimize.brentq}).

\subsection{Coplanarity Correction}

After the joint solve, the reference finger's base-frame $Z$ is
$z^* = I_z(c_\text{ref})$.  Each active non-reference finger $f$ is independently
corrected by finding $c_f$ such that
\begin{equation}
  z_f(c_f) = z^*.
\end{equation}
This is a 1-D root-find on the monotone-decreasing function $z_f(c)$, also solved with
\texttt{brentq}.

If $z^*$ is outside finger $f$'s achievable range $[z_f^{\min},\,z_f^{\max}]$:
\begin{equation}
  c_f = \begin{cases}
    0 & \text{if } z^* \geq z_f^{\max} \quad\text{(target too high, use full extension)}\\
    c_f^{\max} & \text{if } z^* < z_f^{\min} \quad\text{(target too low, use full flex)}
  \end{cases}
\end{equation}
Using the wrong direction (e.g.\ always falling back to $c_f^{\max}$) causes
some fingers to fully close when the reference is near-extended at large widths.

\subsection{Residual World-$Z$ Spread}

Even after coplanar correction, the non-thumb tips show a residual world-$Z$ spread of
$\approx5$~mm.  This arises because coplanarity enforces equal base-frame $Z$, but not
equal base-frame $X$.  With the rotation \eqref{eq:Rmat}:
\begin{equation}
  w_z = -\sin\theta\,p_x - \cos\theta\,p_z.
\end{equation}
Fingers with the same $p_z$ but different $p_x$ land at different $w_z$ after tilt.
This is a fundamental hardware constraint (different finger lengths) and is accepted.

% ---------------------------------------------------------------------------
\section{Cylinder / Power Grasp}

\subsection{Diameter Model}

Power grasps are modelled at the \emph{proximal-intermediate joint} rather than the
fingertip, because the object contacts the proximal links:
\begin{equation}
  D_\text{cyl}(c) = \bigl\|\T_\text{prox}(c) - \bar{\mathbf{p}}_\text{prox}(c)\bigr\|_2,
  \qquad
  \bar{\mathbf{p}}_\text{prox}(c) = \frac{1}{4}\sum_{f} \mathbf{p}_f^{\text{prox}}(c).
\end{equation}
Both the thumb proximal-intermediate joint and the non-thumb centroid are computed via
live \texttt{mj\_kinematics} calls (not interpolated, since they require the full
kinematic chain including body translations).

\subsection{Thumb Yaw Transition}

At large diameters (fingers less closed), the thumb opposes the fingers at
$y = y^{\max} = 1.308\,\text{rad}$.  At small diameters the thumb tip would collide
with the finger tips; below the transition diameter $D^* \approx 71.6\,\text{mm}$ the
thumb switches to palm mode ($y = 0$, $\theta = 0$):
\begin{equation}
  y^* = \begin{cases}
    y^{\max} & D_\text{cyl} \geq D^*\\
    0 & D_\text{cyl} < D^*
  \end{cases}
\end{equation}
$D^*$ is pre-computed at init by sweeping $c$ and finding the first configuration where
the thumb tip approaches within 20~mm of any non-thumb tip.

% ---------------------------------------------------------------------------
\section{World Frame Conversion}

Given a \texttt{ClosureResult} (all positions in base frame, tilt $\theta$):
\begin{align}
  \R &= R_y(\theta)\,R_x(\pi) \\[4pt]
  \bm{m}_w &= \R\,\bm{m}_b \qquad (\text{rotated midpoint}) \\[4pt]
  \bm{b}_w &= \begin{pmatrix}-m_{w,x}\\-m_{w,y}\\g_z - m_{w,z}\end{pmatrix}
  \qquad (\text{hand base in world, midpoint at }g_z) \\[4pt]
  \p_w(f) &= \R\,\p_b(f) + \bm{b}_w \qquad (\text{each tip in world})
\end{align}
$\bm{b}_w$ is always above $g_z$ because $m_{w,z} = -\sin\theta\,m_{b,x} -
\cos\theta\,m_{b,z} \leq 0$ for the achievable closure configurations (the midpoint's
extension-direction component dominates).

% ---------------------------------------------------------------------------
\section{MuJoCo Viewer Rotation Convention}

MuJoCo's free-body rotation joints apply $R_x(r_x)\,R_y(r_y)$ (axes in parent/world
frame).  The FK model uses $\R = R_y(\theta)\,R_x(\pi)$.  Matching these:
\begin{equation}
  R_x(\pi)\,R_y(-\theta) = R_y(\theta)\,R_x(\pi)
  \quad\Longleftrightarrow\quad
  r_x = \pi,\quad r_y = -\theta.
\end{equation}
The negation of $\theta$ for \texttt{rot\_y} is therefore mandatory.

% ---------------------------------------------------------------------------
\section{Verified Width Ranges}

\begin{table}[h]
\centering
\begin{tabular}{lrrp{7cm}}
\toprule
Mode & Min (mm) & Max (mm) & Notes \\
\midrule
2-finger line & 8.7 & 122 & Tilt $\to90°$ (side-approach) below $\approx15.5$~mm \\
3-finger plane & 15.6 & 130 & \\
4-finger plane & 15.6 & 130 & $\approx5$~mm inherent Z-spread at all widths \\
5-finger plane & 15.6 & 130 & \\
cylinder (diameter) & 28.6 & 103 & Palm mode below 71.6~mm \\
\bottomrule
\end{tabular}
\caption{Width/diameter ranges achievable without geometric degeneracy.}
\end{table}

% ---------------------------------------------------------------------------
\section{Key Numerical Values}

\begin{table}[h]
\centering
\begin{tabular}{lrl}
\toprule
Quantity & Value & Source \\
\midrule
$c_\text{pinky}^{\max}$, $c_\text{ring}^{\max}$ & 1.57~rad & XML actuator range \\
$c_\text{middle}^{\max}$, $c_\text{index}^{\max}$ & 1.50~rad & XML actuator range \\
$c_\text{pitch}^{\max}$ & 0.60~rad & XML actuator range \\
$c_\text{yaw}^{\max}$ & 1.308~rad & XML actuator range \\
Coupling preload & 0.15~rad ($\approx8.6°$) & polycoef $a_0$ \\
Finger $Z$-shift (open$\to$closed) & $\approx75$~mm & FK sweep \\
Cylinder yaw transition & 71.6~mm diameter & tip-clearance sweep \\
FK samples (non-thumb) & 200 per finger & \texttt{N\_SAMPLES\_1D} \\
FK samples (thumb) & $50\times50$ & \texttt{N\_SAMPLES\_2D} \\
\bottomrule
\end{tabular}
\end{table}

\end{document}
